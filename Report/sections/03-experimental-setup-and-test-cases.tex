% 03-experimental-setup-and-test-cases.tex
% ---------------------------

% Section Title
\section{EXPERIMENTAL SETUP AND TEST CASES} \label{sec:experimental-setup-and-test-cases}

    % Main Content

    \subsection{Equipment and Configuration} \label{subsec:equipment-and-configuration}

        In this section, we describe the hardware and software configuration used to perform our network performance measurements. 
        Table~\ref{tab:equipment-summary} summarizes the main devices, their interfaces, and relevant specifications. 

        \vspace{-0.1cm} % TODO: check

        \begin{table}[ht]
            \small
            \centering
            \begin{tabular}{@{}l p{0.87\columnwidth}@{}}
            \toprule
            \textbf{Device} & \textbf{Key Specifications} \\
            \midrule
            \textbf{PC1} 
                & \textbf{Victus 16-s1005nl Notebook} \newline
                \textit{Operating System:}   Ubuntu 24.04.2~LTS~\cite{ubuntu} \newline
                \textit{Ethernet Interface:} Realtek RTL8111/8168/8211/8411~\cite{realtek8111} \newline
                \textit{Wireless Interface:} Realtek RTL8852BE (802.11ax) | 2x2~\cite{realtek8852be} \\
            \midrule
            \textbf{PC2} 
                & \textbf{Microsoft Surface Laptop Go~3} \newline
                \textit{Operating System:}   Ubuntu 24.10~\cite{ubuntu} \newline
                \textit{Ethernet Interface:} via Anker PowerExpand+ USB-C Hub~\cite{ankerhub} \newline
                \textit{Wireless Interface:} Intel Alder Lake-P CNVi (802.11ax) | 2x2~\cite{intelcnvi} \\
            \midrule
            \textbf{Router} 
                & \textbf{Vodafone Power Station Wi-Fi~6} \newline
                \textit{Ethernet Ports:}     4 $\times$ 1\,GbE ports \newline
                \textit{Wi-Fi:}              Dual-band 802.11ax (2.4\,GHz | 2x2, 5\,GHz | 4x4)~\cite{vodafonewifi6} \newline
                \textit{Default gateway ip: 192.168.1.1} \\
            \midrule
            \textbf{Cables} 
                & CAT.5E (up to 1\,Gbps) \\
            \bottomrule
            \end{tabular}
            \vspace{0.5cm}
            \caption{Summary of Hardware and Network Configuration}
            \label{tab:equipment-summary}
        \end{table}

        \vspace{-0.6cm} % TODO: check

        \begin{table}[ht]
            \small
            \centering
            \begin{tabular}{@{}l p{0.78\columnwidth}@{}}
            \toprule
            \textbf{Connection} & \textbf{Key Specifications} \\
            \midrule
            \textbf{Ethernet} 
                & \textit{Cabling:}          CAT.5E \newline
                \textit{Nominal Speed:}      1\,Gbps      \hspace{2em}   \textit{Protocol:}           Ethernet II \newline
                \textbf{\textit{Client ip:}} 192.168.1.12 \hspace{1.9em} \textbf{\textit{Server ip:}} 192.168.1.13 \\ 
            \midrule
            \textbf{Wi-Fi} 
                & \textit{Standard:}         802.11ax    \hspace{3.6em}  \textit{Security Protocol:}   WPA2-AES \newline
                \textit{Nominal Speed:}      1200\,Mbps  \hspace{0.5em}  \textit{Frequency:}           5\,GHz \newline
                \textit{Bandwidth:}          80\,MHz     \hspace{3.25em} \textit{Channel:}             100 \newline
                \textbf{\textit{Client ip:}} 192.168.1.8 \hspace{2.4em}   \textbf{\textit{Server ip:}} 192.168.1.4 \newline
                \textbf{\textit{Third host ip:}} 192.168.1.13 \textit{ (shared capacity scenario)} \\
            \bottomrule
            \end{tabular}
            \vspace{0.5cm}
            \caption{Ethernet and Wi-Fi Connection Specifications}
            \label{tab:connection-specs}
        \end{table}

        \vspace{-0.3cm} % TODO: check

        \noindent This hardware setup allows us to compare Ethernet versus Wi-Fi performance under a consistent router and cabling environment. 
        In the next section, we detail the evaluation scenarios and the measurement methodology.

    \subsection{Evaluation Scenarios} \label{subsec:evaluation-scenarios}

        We considered three distinct network configurations to assess the performance differences between wired and wireless communications. 
        For each scenario, the theoretical goodput is computed based on the nominal link capacity and protocol efficiency.

        \begin{enumerate}

            \item \textbf{Both Ethernet:} \\
            In this scenario, both PC1 and PC2 are connected to the router via CAT.5E cables, providing a nominal link capacity of 1\,Gbps. 

            \noindent The efficiency for Ethernet is calculated as follows:
            \[
            \eta_{TCP}^{Eth} \approx \frac{1460}{1460 + 20 + 20 + 38} \approx 94.9\%,
            \]
            \[
            \eta_{UDP}^{Eth} \approx \frac{1472}{1472 + 20 + 8 + 38} \approx 95.7\%.
            \]
            Thus, the expected goodput is:
            \[
            G_{TCP}^{Eth} \leq 0.949 \times 1000\,\text{Mbps} \approx 949\,\text{Mbps},
            \]
            \[
            G_{UDP}^{Eth} \leq 0.957 \times 1000\,\text{Mbps} \approx 957\,\text{Mbps}.
            \]

            \vspace{0.2cm} % TODO: check
            

            \item \textbf{Both Wi-Fi:} \\
            For this configuration, both devices use their wireless interfaces (802.11ax) to connect to the router. 
            Although the nominal Wi-Fi link speed is assumed to be approximately 1.2\,GbEbps, the half-duplex nature of Wi-Fi effectively halves the throughput available for data transfer. 
            Assuming a Wi-Fi efficiency factor of about 80\%, the expected goodput for TCP is:
            \[
            G_{TCP}^{WiFi} \leq 0.80 \times 1.2\,\text{Gbps} \times \frac{1}{2} \approx 480\,\text{Mbps},
            \]
            and similarly for UDP, with a different efficiency factor:
            \[
            G_{UDP}^{WiFi} \leq 0.85 \times 1.2\,\text{Gbps} \times \frac{1}{2} \approx 510\,\text{Mbps}.
            \]

            \vspace{0.1cm} % TODO: check

            \item \textbf{Mixed Scenario:} \\
            In this configuration, one device (PC1) is connected via Ethernet while the other (PC2) uses its Wi-Fi interface. 
            So the expected goodput is determined by the slower link, so we compute the min between the two.
            Since only one side is on Wi-Fi, the Wi-Fi estimated goodput has not to be halved.
            Thus, the theoretical goodput is:

            \vspace{-0.1cm} % TODO: check

            \[
            G_{TCP}^{Mixed} \leq \min \{0.80 \times 1.2\,\text{Gbps}, 0.949 \times 1.0\,\text{Gbps}\} \approx 949\,\text{Mbps},
            \]

            \vspace{-0.2cm} % TODO: check

            \[
            G_{UDP}^{Mixed} \leq \min \{0.85 \times 1.2\,\text{Gbps}, 0.957 \times 1.0\,\text{Gbps}\} \approx 957\,\text{Mbps}.
            \]

            \vspace{0.1cm} % TODO: check

            We can see that the Eth. link is the bottleneck in this scenario.

        \end{enumerate}

        \vspace{0.1cm} % TODO: check

        \noindent These calculations provide the theoretical upper bounds for goodput in each scenario. 
        The experimental results, obtained via automated measurements using the provided Python script, are compared against these predictions to evaluate real-world performance.

        % \medskip
