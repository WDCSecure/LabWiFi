% 02-methodology-and-concepts.tex
% ---------------------------

% Section Title
\section{METHODOLOGY AND CONCEPTS} \label{sec:methodology-and-concepts}

    % Main Content

    This section outlines the experimental setup, the tools employed for the measurements, and the theoretical basis for estimating goodput.

    \subsection{Selected Tools} \label{subsec:selected-tools}

        To evaluate the performance of both Ethernet and WiFi connections, we utilized several specialized tools:

        \begin{itemize}
            \item \textbf{iperf3}: Used to generate traffic and measure goodput in both TCP and UDP modes. By executing repeated tests, iperf3 provides key metrics such as minimum, maximum, average, and standard deviation of the throughput.
            \item \textbf{Wireshark}: Employed to capture and analyze network traffic, Wireshark enabled us to inspect data flows, identify control and data frames, and validate experimental results.
            \item \textbf{Automation Script}: A Python script was developed to automate the entire measurement process. This script manages both server and client modes of iperf3, logs output in JSON, CSV, and plain text formats, and computes summary statistics. 
                The script accepts several command-line flags, as detailed in the Appendix. % TODO: review
            % \begin{itemize}
            %     \item \textbf{--server}: Launches the iperf3 server in JSON output mode.
            %     \item \textbf{<SERVER\_IP>}: Specifies the server's IP address when running in client mode.
            %     \item \textbf{--udp}: Switches the test from the default TCP mode to UDP.
            %     \item \textbf{--bitrate}: Sets the target bitrate for UDP tests (e.g., \texttt{10M} for 10 Mbps).
            %     \item \textbf{--iter}: Determines the number of test iterations to perform.
            % \end{itemize}
            % The output files (logs, JSON, CSV reports, and raw output) generated by this script are used to document the experimental results and facilitate further analysis.
        \end{itemize}

    \subsection{Goodput Estimation} \label{subsec:goodput-estimation}

        Goodput represents the rate at which useful data is delivered to the application layer, excluding protocol overheads and retransmitted packets. 
        The theoretical estimation of goodput is based on the efficiency of the protocol and the capacity of the network link:
        \[
        G \leq \eta_{\text{protocol}} \times C,
        \]
        where \(C\) is the capacity of the bottleneck link and \(\eta_{\text{protocol}}\) is the protocol efficiency.

        \begin{enumerate}

            \item 
                For \textbf{Ethernet}, the efficiency for TCP is computed as:
                \[
                \eta_{TCP}^{Eth} = \frac{MSS}{MSS + \text{TCP headers} + \text{IP headers} + \text{Eth. overhead}},
                \]
                with the Maximum Segment Size (MSS) defined as the MTU minus the headers. For a standard MTU of 1500 bytes, we obtain:

                \begin{itemize}
                    \item MSS \(\approx\) 1460 bytes (after subtracting 20 bytes for the IP header and 20 bytes for the TCP header),
                    \item An additional Ethernet overhead of approximately 38 bytes.
                \end{itemize}

                \noindent Thus, the efficiency for TCP over Ethernet is approximately:
                \[
                \eta_{TCP}^{Eth} \approx \frac{1460}{1460 + 20 + 20 + 38} \approx 94.9\%.
                \]

                \noindent Similarly, the efficiency for UDP is computed as follows. Since UDP has an 8-byte header, its MSS is given by:

                \begin{itemize}
                    \item MSS \(\approx\) 1472 bytes (after subtracting 20 bytes for the IP header and 8 bytes for the UDP header).
                \end{itemize}

                \noindent Thus, the efficiency for UDP over Ethernet is given by:
                \[
                \eta_{UDP}^{Eth} \approx \frac{1472}{1472 + 20 + 8 + 38} \approx 95.7\%.
                \]

            \item 
                For \textbf{WiFi}, additional factors must be considered due to its half-duplex nature and the inherent overhead of the 802.11 protocol (e.g., control frames, retransmissions, and channel contention). 
                Consequently, the effective efficiency is reduced by a WiFi-specific factor (\(\eta_{WiFi}\)). 
                The adjusted efficiency for TCP over WiFi can be expressed as:
                \[
                \eta_{TCP}^{WiFi} = \eta_{TCP}^{Eth} \times \eta_{WiFi},
                \]
                with \(\eta_{WiFi}\) typically around 80\% in optimal conditions. A similar adjustment applies for UDP.

        \end{enumerate}

        \noindent These theoretical estimates set an upper bound on the achievable goodput, against which our experimental results are compared. 
        Discrepancies between theory and practice are primarily due to dynamic environmental factors, such as interference, channel variability, and the inherent limitations of wireless communication.
