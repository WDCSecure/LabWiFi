% 02-methodology-and-concepts.tex
% ---------------------------

% Section Title
\section{METHODOLOGY AND CONCEPTS} \label{sec:methodology-and-concepts}

    % Main Content

    This section outlines the experimental setup, the tools employed for the measurements, and the theoretical basis for estimating goodput.

    \subsection{Selected Tools} \label{subsec:selected-tools}

    To evaluate the performance of Ethernet and WiFi connections, we utilized several specialized tools:

        \begin{itemize}
            \item \textbf{iperf3}: Used to generate traffic and measure goodput in both TCP and UDP modes. With repeated trials, iperf3 provides valuable metrics such as minimum, maximum, average, and standard deviation of the throughput.
            \item \textbf{Wireshark}: For capturing and analyzing network traffic, Wireshark helped analyze data flows, identify frames, and confirm results. It also generated useful charts for analyzing TCP streams.
            \item \textbf{Automation Script}: A Python script was written to automate the entire measurement process. The script operates on both server and client modes of iperf3, logs output, and computes summary statistics. 
                The script accepts a series of command-line flags, as described in the Appendix~\ref{sec:appendix}.
            % \begin{itemize}
            %     \item \textbf{--server}: Launches the iperf3 server in JSON output mode.
            %     \item \textbf{<SERVER\_IP>}: Specifies the server's IP address when running in client mode.
            %     \item \textbf{--udp}: Switches the test from the default TCP mode to UDP.
            %     \item \textbf{--bitrate}: Sets the target bitrate for UDP tests (e.g., \texttt{10M} for 10 Mbps).
            %     \item \textbf{--iter}: Determines the number of test iterations to perform.
            % \end{itemize}
            % The output files (logs, JSON, CSV reports, and raw output) generated by this script are used to document the experimental results and facilitate further analysis.
        \end{itemize}

    \subsection{Goodput Estimation} \label{subsec:goodput-estimation}

        Goodput represents the rate at which useful data is delivered to the application layer, excluding protocol overheads and retransmitted packets. 
        The theoretical estimation of goodput is based on the efficiency of the protocol and the capacity of the network link:
        \[
        G \leq \eta_{\text{protocol}} \times C,
        \]
        where \(C\) is the capacity of the bottleneck link and \(\eta_{\text{protocol}}\) is the protocol efficiency.

        \begin{enumerate}

            \item 
                For \textbf{Ethernet}~\cite{ieee8023}, the efficiency for TCP is computed as:
                \[
                \eta_{TCP}^{Eth} = \frac{MSS}{MSS + \text{TCP headers} + \text{IP headers} + \text{Eth. overhead}},
                \]
                with the Maximum Segment Size (MSS) defined as the MTU minus the headers. For a standard MTU of 1500 bytes, we obtain:

                \begin{itemize}
                    \item MSS \(\approx\) 1460 bytes (after subtracting 20 bytes for the IP header and 20 bytes for the TCP header),
                    \item An additional Ethernet overhead of approximately 38 bytes.
                \end{itemize}

                \noindent Thus, the efficiency for TCP over Ethernet is approximately:
                \[
                \eta_{TCP}^{Eth} \approx \frac{1460}{1460 + 20 + 20 + 38} \approx 94.9\%.
                \]

                \noindent Similarly, the efficiency for UDP is computed as follows. Since UDP has an 8-byte header, its MSS is given by:

                \begin{itemize}
                    \item MSS \(\approx\) 1472 bytes (after subtracting 20 bytes for the IP header and 8 bytes for the UDP header).
                \end{itemize}

                \noindent Thus, the efficiency for UDP over Ethernet is given by:
                \[
                \eta_{UDP}^{Eth} \approx \frac{1472}{1472 + 20 + 8 + 38} \approx 95.7\%.
                \]


            \item  
                For \textbf{WiFi}~\cite{ieee80211ax}, there are additional considerations to account for due to its half-duplex nature and the overhead of the 802.11 protocol (e.g., control frames, retransmissions, and channel contention). 
                For \textbf{TCP over WiFi}, the actual efficiency is typically around 80\% under optimal conditions. 
                This lower efficiency arises from the extra overhead of TCP’s connection-oriented features—such as congestion control, 
                flow control, and the guarantee of in-order delivery—which add depending on extra control packets and retransmissions.
                
                Whereas, \textbf{UDP over WiFi} typically achieves the efficiency of around 85–90\% by avoiding these mechanisms, 
                leading to a simpler and faster data transmission process.
        
                \[
                    \eta_{TCP}^{WiFi} (\approx 80\%) \hspace{1em} \text{ and } \hspace{1em} \eta_{UDP}^{WiFi} (\approx 85\text{–}90\%).
                \]
            
        \end{enumerate}

        \noindent These theoretical estimates impose an upper bound on the amount of achievable goodput that our results are compared to.
        Discrepancies between theory and practice are primarily due to dynamic environmental variables, such as interference, channel variation, and nature's intrinsic constraint on wireless communications.
