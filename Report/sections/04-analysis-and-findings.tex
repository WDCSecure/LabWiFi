% 04-analysis-and-findings.tex
% ---------------------------

\section{ANALYSIS AND FINDINGS} \label{sec:analysis-and-findings}

    \subsection{TCP Performance} \label{subsec:tcp-performance}

        The performance tests using TCP reveal several noteworthy trends. The analysis for TCP performance in different scenarios is organized as follows:

        \begin{table}[H]
            \small
            \centering
            \begin{tabular}{|ll|lllll|}
            \hline
            % \multicolumn{2}{|c|}{\multirow{2}{*}{\makecell{\textbf{Test} \\ Client $\rightarrow$ Server}}} & 
            \multicolumn{2}{|c|}{\multirow{2}{*}{\textbf{Test}}} & 
                \multicolumn{5}{c|}{\textbf{TCP: Goodput per flow (Mbps)}} \\
            \cline{3-7}
            \multicolumn{2}{|c|}{} &
                \multicolumn{1}{c|}{Prediction} &
                \multicolumn{1}{c|}{Average} &
                \multicolumn{1}{c|}{Min} &
                \multicolumn{1}{c|}{Max} &
                \multicolumn{1}{c|}{Std} \\
            \hline
            \multicolumn{2}{|c|}{Both Ethernet} &
                \multicolumn{1}{c|}{949} &
                \multicolumn{1}{c|}{939.6} &
                \multicolumn{1}{c|}{938.2} &
                \multicolumn{1}{c|}{942.7} &
                \multicolumn{1}{c|}{1.5} \\
            \hline
            \multicolumn{2}{|c|}{Both WiFi} &
                \multicolumn{1}{c|}{480} &
                \multicolumn{1}{c|}{434.7} &
                \multicolumn{1}{c|}{396.3} &
                \multicolumn{1}{c|}{461.96} &
                \multicolumn{1}{c|}{22.5} \\
            \hline
            \multicolumn{2}{|c|}{Mixed} &
                \multicolumn{1}{c|}{949} &
                \multicolumn{1}{c|}{663.7} &
                \multicolumn{1}{c|}{619.2} &
                \multicolumn{1}{c|}{698.86} &
                \multicolumn{1}{c|}{26.6} \\
            \hline
           % \multicolumn{2}{|c|}{Shared Capacity} &
           %     \multicolumn{1}{c|}{?} &
           %     \multicolumn{1}{c|}{536.8} &
           %     \multicolumn{1}{c|}{440.5} &
           %     \multicolumn{1}{c|}{722.2} &
           %     \multicolumn{1}{c|}{112} \\
           % \hline
            \end{tabular}
            \vspace{0.5cm}
            \caption{TCP Results (Client $\rightarrow$ Server)}
            \label{tab:tcp-results}
        \end{table}

        \begin{enumerate}

            \item \textbf{Both Ethernet:} \\
                Figure~\ref{fig:throughput-eth-tcp} shows the TCP throughput measured in the Ethernet scenario. 
                The graph reveals a rapid ramp-up in throughput during the first few seconds, followed by a stable transmission rate that approaches the theoretical value. 
                The \textbf{Maximum Segment Size} (MSS) reaches and remains stable at 1500 bytes, as defined by the TCP protocol. Additionally, the \textbf{bandwidth} is stable at \textbf{950 Mbps}, as indicated by the results and the low standard deviation.              
                
                \vspace{-0.1cm} % TODO: check

                \begin{figure}[ht]
                    \centering
                    \includegraphics[width=0.9\columnwidth]{images/graphs/Throughput/Throughput_ETH_TCP.pdf}
                    \caption{TCP Throughput in the Ethernet Scenario.}
                    \label{fig:throughput-eth-tcp}
                \end{figure}

                Figure~\ref{fig:rtt-eth-tcp} illustrates the round-trip time (RTT), which remains very low (typically within 1-3 milliseconds), highlighting the minimal latency in wired connections. 
                
                \begin{figure}[ht]
                    \centering
                    \includegraphics[width=0.9\columnwidth]{images/graphs/RTT/RTT_ETH_TCP.pdf}
                    \caption{TCP Round Trip Time in the Ethernet Scenario.}
                    \label{fig:rtt-eth-tcp}
                \end{figure}

                % Furthermore, the I-O graph (Fig.~\ref{fig:io-eth-tcp}) confirms a consistent packet flow with little variation, indicating that the Ethernet setup effectively utilizes the available capacity. 
                
                % \begin{figure}[ht]
                %     \centering
                %     \includegraphics[width=0.9\columnwidth]{images/graphs/I-O/I-O_ETH_TCP.pdf}
                %     \caption{Wireshark I-O Graph for TCP in the Ethernet Scenario.}
                %     \label{fig:io-eth-tcp}
                % \end{figure}

                Overall, the Ethernet scenario demonstrates a near-ideal performance with high throughput and minimal latency, closely matching the theoretical predictions.

            \vspace{0.2cm} % TODO: check

            \item \textbf{Both WiFi:} \\
                In the WiFi scenario, the throughput graph (Fig.~\ref{fig:throughput-wifi-tcp}) shows an initial ramp-up phase during the first 2 seconds, after which the throughput fluctuates around an average value of 434.7\,Mbps. 
                These fluctuations suggest that protocol overhead, wireless interference, and the half-duplex nature of WiFi adversely affect performance.
                
                \begin{figure}[ht]
                    \centering
                    \includegraphics[width=0.9\columnwidth]{images/graphs/Throughput/Throughput_WiFi_TCP.pdf}
                    \caption{TCP Throughput in the WiFi Scenario. \vspace{0.2cm}} % TODO: check
                    \label{fig:throughput-wifi-tcp}
                \end{figure}

                The round-trip time (RTT) measurements (Fig.~\ref{fig:rtt-wifi-tcp}) reveal RTT values ranging from about 20-50\,ms, indicating intermittent delays likely due to congestion and contention in the wireless medium.
                
                \begin{figure}[ht]
                    \centering
                    \includegraphics[width=0.9\columnwidth]{images/graphs/RTT/RTT_WiFi_TCP.pdf}
                    \caption{TCP Round Trip Time in the WiFi Scenario.}
                    \label{fig:rtt-wifi-tcp}
                \end{figure}    

                % Furthermore, the I-O graph (Fig.~\ref{fig:io-wifi-tcp}) illustrates a variable number of transmitted packets per interval, reflecting the dynamic nature of WiFi communication where channel conditions and collision avoidance mechanisms influence performance.
                
                % \begin{figure}[ht]
                %     \centering
                %     \includegraphics[width=0.9\columnwidth]{images/graphs/I-O/I-O_WiFi_TCP.pdf}
                %     \caption{Wireshark I-O Graph for TCP in the WiFi Scenario.}
                %     \label{fig:io-wifi-tcp}
                % \end{figure}}

                Overall, while the theoretical capacity for TCP over WiFi is estimated to be around 480\,Mbps, the experimental data indicate that real-world factors does not reduce that much the effective throughput of the WiFi network. 
                This due to the fact that the network was in an ideal condition, where only the client and the server were connected to the access point, and no other devices were generating traffic.

            \vspace{0.2cm} % TODO: check

            \item \textbf{Mixed:} \\
                Figure~\ref{fig:throughput-mix-tcp} displays the TCP throughput for the mixed configuration. 
                The graph shows that the throughput reaches a stable level after an initial ramp-up phase, although it remains below the Ethernet scenario and is consistent with the expected reduction due to the reliance on the wireless link.

                \begin{figure}[ht]
                    \centering
                    \includegraphics[width=0.9\columnwidth]{images/graphs/Throughput/Throughput_MIX_TCP.pdf}
                    \caption{TCP Throughput in the Mixed Ethernet/WiFi Scenario.}
                    \label{fig:throughput-mix-tcp}
                \end{figure}

                The round-trip time (RTT) measurements, presented in Figure~\ref{fig:rtt-mix-tcp}, indicate moderate latency, with RTT values generally remaining within a lower range compared to the pure WiFi scenario. 
                This suggests that the wired segment helps in reducing overall latency.

                \begin{figure}[ht]
                    \centering
                    \includegraphics[width=0.9\columnwidth]{images/graphs/RTT/RTT_MIX_TCP.pdf}
                    \caption{TCP Round Trip Time in the Mixed Ethernet/WiFi Scenario.}
                    \label{fig:rtt-mix-tcp}
                \end{figure}

                % The I-O graph for TCP (Fig.~\ref{fig:io-mix-tcp}) shows a relatively steady packet flow over the test intervals, confirming that the mixed configuration maintains a stable performance despite the inherent variability of the wireless link.

                % \begin{figure}[ht]
                %     \centering
                %     \includegraphics[width=0.9\columnwidth]{images/graphs/I-O/I-O_MIX_TCP.pdf}
                %     \caption{Wireshark I-O Graph for TCP in the Mixed Scenario.}
                %     \label{fig:io-mix-tcp}
                % \end{figure}

            \vspace{0.2cm} % TODO: check
                
            \item[3a.] \textbf{Shared Capacity:} \\
                In this scenario, a third host connected to the same access point was concurrently downloading the film "Natale a Rio"~\cite{Natale_a_Rio} (directed by Neri Parenti), which introduced significant interference during the tests. 
                This additional traffic compromised the available network capacity, leading to degraded performance.
                % Figure~\ref{fig:throughput-mitm-tcp} shows the TCP throughput under this shared capacity condition. 
                % Compared to the mixed scenario without interference, the throughput exhibits a notable decrease. 
                As we can see in (Fig.~\ref{fig:io-mitm-tcp}), the other host starts the download from the second test (\textasciitilde25 seconds). In fact the value of the throughtput, goes from the one of the standard Mixed Scenario (\textbf{\textasciitilde670Mbps}), to a lower one (\textbf{\textasciitilde500Mbps}),
                This behavior is due to the fact that the bandwidth is shared between the two hosts, and the download of the movie is consuming useful bandwidth.

                % \begin{figure}[ht]
                %     \centering
                %     \includegraphics[width=0.9\columnwidth]{images/graphs/Throughput/Throughput_MIX_MITM_TCP.pdf}
                %     \caption{TCP Throughput in the Shared Capacity Scenario (with third-host interference).}
                %     \label{fig:throughput-mitm-tcp}
                % \end{figure}

                % The round-trip time measurements (Fig.~\ref{fig:rtt-mitm-tcp}) indicate increased variability and slightly elevated latency. 
                % Although the RTT values remain relatively moderate, the fluctuations suggest that the network experiences occasional congestion and delays as a result of the third host's activity.

                % \begin{figure}[ht]
                %     \centering
                %     \includegraphics[width=0.9\columnwidth]{images/graphs/RTT/RTT_MIX_MITM_TCP.pdf}
                %     \caption{TCP Round Trip Time in the Shared Capacity Scenario.}
                %     \label{fig:rtt-mitm-tcp}
                % \end{figure}

                % The I-O graph for TCP (Fig.~\ref{fig:io-mitm-tcp}) further confirms the impact of the interference. 
                % The graph displays irregular intervals and a lower packet transmission rate compared to the mixed scenario without the additional load, demonstrating how the extra traffic disrupts the steady flow of data.

                \begin{figure}[ht]
                    \centering
                    \includegraphics[width=0.9\columnwidth]{images/graphs/I-O/I-O_MIX_MITM_TCP.pdf}
                    \caption{Wireshark I-O Graph for TCP in the Shared Capacity Scenario.}
                    \label{fig:io-mitm-tcp}
                \end{figure}

        \end{enumerate}

    \subsection{UDP Performance} \label{subsec:udp-performance}

        The UDP tests offer an insightful comparison to the TCP results by eliminating congestion control and acknowledgment overhead. The analysis for UDP is structured as follows:

        \begin{table}[H]
            \small
            \centering
            \begin{tabular}{|ll|lllll|}
            \hline
            % \multicolumn{2}{|c|}{\multirow{2}{*}{\makecell{\textbf{Test} \\ Client $\rightarrow$ Server}}} & 
            \multicolumn{2}{|c|}{\multirow{2}{*}{\textbf{Test}}} & 
                \multicolumn{5}{c|}{\textbf{UDP: Goodput per flow (Mbps)}} \\
            \cline{3-7}
            \multicolumn{2}{|c|}{} &
                \multicolumn{1}{c|}{Prediction} &
                \multicolumn{1}{c|}{Average} &
                \multicolumn{1}{c|}{Min} &
                \multicolumn{1}{c|}{Max} &
                \multicolumn{1}{c|}{Std} \\
            \hline
            \multicolumn{2}{|c|}{Both Ethernet} &
                \multicolumn{1}{c|}{957} &
                \multicolumn{1}{c|}{952.8} &
                \multicolumn{1}{c|}{948.3} &
                \multicolumn{1}{c|}{954.6} &
                \multicolumn{1}{c|}{1.73} \\
            \hline
            \multicolumn{2}{|c|}{Both WiFi} &
                \multicolumn{1}{c|}{510} &
                \multicolumn{1}{c|}{487.8} &
                \multicolumn{1}{c|}{453.1} &
                \multicolumn{1}{c|}{499.9} &
                \multicolumn{1}{c|}{15.8} \\
            \hline
            \multicolumn{2}{|c|}{Mixed} &
                \multicolumn{1}{c|}{957} &
                \multicolumn{1}{c|}{674.9} &
                \multicolumn{1}{c|}{636.6} &
                \multicolumn{1}{c|}{717.8} &
                \multicolumn{1}{c|}{28} \\
            \hline
           % \multicolumn{2}{|c|}{Shared Capacity} &
           %     \multicolumn{1}{c|}{?} &
           %     \multicolumn{1}{c|}{472.1} &
           %     \multicolumn{1}{c|}{355.9} &
           %     \multicolumn{1}{c|}{699.1} &
           %     \multicolumn{1}{c|}{121.1} \\
           % \hline
            \end{tabular}
            \vspace{0.5cm}
            \caption{UDP Results (Client $\rightarrow$ Server)}
            \label{tab:udp-results}
        \end{table}


        \begin{enumerate}
            \item \textbf{Both Ethernet:} \\
                In the Ethernet scenario, UDP achieves a near-theoretical throughput of \textbf{952.8 Mbps} (vs. TCP’s 939.6 Mbps), with minimal standard deviation (\textbf{1.73 Mbps}). The absence of retransmissions or congestion control allows UDP to utilize the full wired capacity. While TCP’s latency remains marginally lower (1–3 ms RTT) due to acknowledgment-based stability, UDP’s lack of overhead enables slightly higher throughput.
            \\

            \item \textbf{Both WiFi:} \\
            UDP averages \textbf{487.8 Mbps} (vs. TCP’s 434.7 Mbps) in WiFi, achieving a \textbf{\textasciitilde 12\% throughput advantage} by avoiding TCP’s congestion control. Despite outperforming TCP, UDP falls short of the \textbf{510 Mbps theoretical maximum} due to WiFi interference and contention. Moreover, it shows lower variability (std. dev. \textbf{15.8 Mbps} vs. TCP’s \textbf{22.5 Mbps}), indicating smoother performance. This makes UDP preferable for real-time applications prioritizing speed over error correction.
            \\

            \item \textbf{Mixed:} \\
                The mixed scenario shows UDP achieving \textbf{674.9 Mbps} (vs. TCP’s 663.7 Mbps). The wired segment reduces latency, but the wireless link remains the bottleneck. UDP’s performance aligns closely with TCP here, as both protocols are constrained by WiFi’s limitations. UDP’s throughput is \textbf{1.8\% higher} than TCP, reflecting its ability to bypass congestion control.
            \\

            \item[3a.] \textbf{Shared Capacity:} \\
                In the shared capacity test, UDP throughput drops significantly. The third host’s download introduces contention, reducing available bandwidth. UDP’s lack of congestion control leads to aggressive transmission attempts, but packet drops result in lower effective throughput. Unlike TCP, which degrades predictably due to its back-off mechanism, UDP’s performance becomes more volatile. This highlights TCP’s adaptability in shared environments, where fairness and resource allocation are critical, while UDP’s rigidity makes it less suitable for contested networks.
        \end{enumerate}
    