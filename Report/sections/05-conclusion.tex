\section{CONCLUSION} \label{sec:conclusion}

    This research compared TCP and UDP performance over wired, wireless, and hybrid networks. 
    Results affirm that \textbf{Ethernet environments} optimize protocol efficiency, achieving near-theoretical throughput with negligible latency. 
    \textbf{WiFi}, on the other hand, en-troduces variability because of interference and contention tha taffects both protocols in spite of controlled scenarios.
    
    \noindent
    \textbf{TCP} prioritizes reliability, making it resilient for file transfers and web traffic, though its congestion control limits performance in dynamic environments. 
    \textbf{UDP}, although quicker in ideal circumstances, struggles with packet loss and shared resources, making it less suitable for congested networks.

    \noindent
    Hybrid \textbf{Ethernet-WiFi setups} focus attention on the ethernet portion as the primary bottleneck, with both protocols also constrained by WiFi’s instability. 
    Shared capacity configurations further reduce performance, underlining the requirement for adaptive protocols in contested networks.

    \noindent
    These results suggest the importance of aligning protocol selection with environmental constraints. 
    Although theoretical models provide benchmarks, real-world performance hinges on interference, medium contention, and protocol design. 
    Network planning must balance throughput, latency, and resilience against dynamic conditions.

        
