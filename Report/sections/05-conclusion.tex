\section{CONCLUSION} \label{sec:conclusion}

    This study evaluated TCP and UDP performance across wired, wireless, and hybrid networks. 
    Results confirm that \textbf{Ethernet environments} maximize protocol efficiency, enabling near-theoretical throughput with minimal latency. 
    In contrast, \textbf{WiFi introduces variability} due to interference and contention, impacting both protocols despite controlled conditions.

    \noindent
    \textbf{TCP} prioritizes reliability, making it robust for file transfers and web traffic, though its congestion control limits performance in dynamic environments. 
    \textbf{UDP}, while faster in ideal conditions, struggles with packet loss and shared resources, rendering it less suitable for congested networks.

    \noindent
    Hybrid \textbf{Ethernet-WiFi setups} highlight the wireless segment as the primary bottleneck, with both protocols constrained by WiFi’s instability. 
    Shared capacity scenarios further degrade performance, emphasizing the need for adaptive protocols in contested networks.

    \noindent
    These findings underscore the importance of aligning protocol choice with environmental constraints. 
    While theoretical models provide benchmarks, real-world performance hinges on interference, medium contention, and protocol design. 
    Network planning must balance throughput, latency, and resilience to dynamic conditions.
