% main.tex

% Uncomment the following line if you need to adjust title spacing.
% \newcommand{\compacttitlespacing}{0} % disable when we need room for authors

% ----------------------------------------
% Document Class and Top Matter Settings
% ----------------------------------------
% \documentclass[sigconf, review]{template/acmart}
\documentclass[sigconf]{template/acmart}

% Remove ACM-specific references and permissions
\setcopyright{none} % Disable copyright
\makeatletter
\@printpermissionfalse
\@printcopyrightfalse
\@acmownedfalse
\makeatother

\acmConference{Wireless Security Report}{Torino}{2024} % set a possibly significant conference name
\acmPrice{} % leave the rest empty. These will still appear.
\acmISBN{}
\acmDOI{}

% Remove ACM reference format
\settopmatter{printacmref=false} % Disable ACM reference format
\renewcommand\footnotetextcopyrightpermission[1]{}

% ----------------------------------------
% Custom Definition for \BibTeX (Optional)
% ----------------------------------------
% The following defines the \BibTeX command in a specific style based on Oren Patashnik's original documentation.
% Uncomment these lines if you want to use this custom formatting for \BibTeX.
% defining the \BibTeX command - from Oren Patashnik's original BibTeX documentation.
\def\BibTeX{{\rm B\kern-.05em{\sc i\kern-.025em b}\kern-.08emT\kern-.1667em\lower.7ex\hbox{E}\kern-.125emX}}
    
% ----------------------------------------
% Packages for Numbers, Units, and Tables
% ----------------------------------------
\usepackage{nicefrac}
\usepackage{siunitx}
\usepackage{array,framed}
\usepackage{booktabs}
\usepackage{textcomp}

% ----------------------------------------
% Graphics and Color Packages
% ----------------------------------------
\usepackage{
  color,
  float,
  epsfig,
  wrapfig,
  graphics,
  graphicx,
  subcaption
}
% Uncomment the following line if you need additional named colors.
% \usepackage[dvipsnames]{xcolor}

% ----------------------------------------
% Packages for Symbols and Fonts
% ----------------------------------------
\let\Bbbk\relax  % Relax \Bbbk if already defined (e.g., by acmart)
\usepackage{textcomp,amssymb}
\usepackage{setspace}
% Uncomment the following line if you require additional AMS fonts.
% \usepackage{amsfonts}
\usepackage{latexsym,fancyhdr,url}

\newcommand{\cooltext}[1]{%
  \texttt{\colorbox{gray!15}{\textcolor{black}{#1}}}%
}

% ----------------------------------------
% Packages for Lists, Algorithms, and Parsing
% ----------------------------------------
\usepackage{enumerate}
\usepackage{algorithm2e}
\usepackage{algpseudocode}
\usepackage{graphics} % (Note: duplicate; consider removing if not needed)
\usepackage{xparse}   % For advanced command parsing (e.g., \edist)
\usepackage{xspace}
\usepackage{multirow}
\usepackage{makecell}
\usepackage{csvsimple}
\usepackage{balance}
\usepackage{listings}
\usepackage{enumitem}
% Uncomment the following line if you need to balance columns in two-column mode.
% \usepackage{flushend}

% Custom settings for listings
\lstset{
    language=Python,                    % Sets the programming language for syntax highlighting
    basicstyle=\footnotesize\ttfamily,  % Sets the font style and size for the code
    keywordstyle=\color{blue},          % Style applied to keywords (e.g., "def", "import")
    commentstyle=\color{gray},          % Style applied to comments (e.g., lines starting with "#")
    stringstyle=\color{red},            % Style applied to strings (e.g., text within quotes)
    % backgroundcolor=\color{gray!10},  % Background color for the code block (light gray in this case)
    % numbers=left,                     % Displays line numbers on the left side of the code block
    % numberstyle=\tiny,                % Font size/style for line numbers
    % stepnumber=1,                     % Line number increment (e.g., every 1 line gets a number)
    % numbersep=5pt,                    % Distance between the line numbers and the code
    linewidth=0.95\linewidth,           % Code block occupies 80% of text width
    xleftmargin=-20pt,                  % Moves the listing 20pt closer to the left margin
    xrightmargin=5pt,                   % Optional: Adds extra padding on the right
    breaklines=true,                    % Automatically breaks long lines to fit within the page width
    captionpos=b,                       % Places the caption below the code block
    abovecaptionskip=5pt,               % Adjusts the space above the caption to 5pt
    belowcaptionskip=8pt,               % Adjusts the space below the caption to 8pt
}

% Configure bullet points with customized size
\setlist[itemize,1]{label={\Large\textbullet}, leftmargin=1em} % First level: larger bullet point
\setlist[itemize,2]{label={\Large\textbullet}, leftmargin=1em} % Second level: larger bullet point and indented by 1em

% Configure enumerate lists with customized size
\setlist[enumerate,1]{label=\arabic*., leftmargin=1.5em} % First level: indented by 1em
\setlist[enumerate,2]{label=\arabic*., leftmargin=1em}   % Second level: indented by 1em

% ----------------------------------------
% Optional Font Packages
% ----------------------------------------
% Uncomment the following line to use Times and Avant fonts.
% \usepackage{mathptmx,avant}

% ----------------------------------------
% TikZ and PGFPlots Packages
% ----------------------------------------
\usepackage{
  tikz,
  pgfplots,
  pgfplotstable
}
\usepackage{hyperref} % For hyperlinks

\usetikzlibrary{
  shapes.geometric,
  arrows,
  external,
  pgfplots.groupplots,
  matrix
}

\pgfplotsset{compat=1.9}
% Uncomment the following lines to externalize TikZ images (improves compilation speed in complex documents).
% \tikzexternalize[prefix=images/]
% \tikzexternalenable

% ----------------------------------------
% Page Numbering and Style (Optional)
% ----------------------------------------
% Uncomment these lines to set page numbering and page style if desired.
% \pagenumbering{arabic}
% \pagestyle{plain}

% Trying to force the review numbering
\usepackage[switch]{lineno} % TODO: check
\renewcommand\linenumberfont{\normalfont\small\color{black}} % TODO: check color

% ----------------------------------------
% Math Packages and Definitions
% ----------------------------------------
\usepackage{mathtools,}
\DeclarePairedDelimiter\abs{\lvert}{\rvert}
\DeclarePairedDelimiter\norm{\lVert}{\rVert}

% Uncomment the following line to set a specific math font.
% \setmathfont{Latin Modern Math}[version=lm]
\DeclareMathAlphabet{\mathcal}{OMS}{cmsy}{m}{n}
% The following math alphabet declarations are commented out by default.
% Uncomment them if you need to redefine these symbols.
% \DeclareSymbolFont{operators}{T1}{cmr}{m}{n}
% \DeclareSymbolFont{letters}{OML}{cmm}{m}{it}
% \DeclareSymbolFont{symbols}{OMS}{cmsy}{m}{n}
% \DeclareSymbolFont{largesymbols}{OMX}{cmex}{m}{n}

% Optional font settings:
% Uncomment the following lines if you prefer using Times font or setting the mathcal font to Arial.
% \usepackage{times}
% \setmathcal{Arial}

% ----------------------------------------
% Page Layout Adjustments (Optional)
% ----------------------------------------
% Uncomment and adjust these settings if you need to control orphans/widows.
% \brokenpenalty=1000
% \clubpenalty=1000
% \widowpenalty=10

% ----------------------------------------
% Graphics Extensions and Custom Definitions
% ----------------------------------------
\DeclareGraphicsExtensions{%
    .png,.PNG,%
    .pdf,.PDF,%
    .jpg,.mps,.jpeg,.jbig2,.jb2,%
    .JPG,.JPEG,.JBIG2,.JB2}
    
% Input custom definitions from defs.tex.
% If you move defs.tex to a different directory, update the path accordingly.
\input{template/defs}

% Adjust caption and footer spacing
\setlength{\belowcaptionskip}{-10pt} 
\setlength{\footskip}{30pt}
\setlength{\abovecaptionskip}{5pt plus 3pt minus 2pt} 

%%%%%%%%%%%%%%%%%%%%%%%%%%%%%%%%%%%%%%%%%%%%%%%%%%%%%%%%%%%%%%%%%%%%%%%%%%%%%%
%                               Begin Document                               %
%%%%%%%%%%%%%%%%%%%%%%%%%%%%%%%%%%%%%%%%%%%%%%%%%%%%%%%%%%%%%%%%%%%%%%%%%%%%%%

\begin{document}

% Optional: Set the font family for the document text.
% Uncomment the following line if you want to change the document font.
% \fontfamily{lmr}\selectfont

% Clear header settings (from fancyhdr)
\fancyhead{}

% ---------------------------
% Document Title
% ---------------------------
\def\thetitle{Performance Evaluation in Ethernet and WiFi Scenarios}
\title{\thetitle}

% ---------------------------
% Authors
% ---------------------------
\author{Andrea Botticella}
\authornote{
  The authors collaborated closely in developing this project.
}
\authornote{
  All the authors are students at Politecnico di Torino, Turin, Italy.
}
\email{andrea.botticella@studenti.polito.it}
\affiliation{%
  % \institution{Politecnico di Torino}
  % \city{Turin}
  % \country{Italy}
}

\author{Elia Innocenti}
\authornotemark[1]
\authornotemark[2]
\email{elia.innocenti@studenti.polito.it}
\affiliation{%
  % \institution{Politecnico di Torino}
  % \city{Turin}
  % \country{Italy}
}

\author{Renato Mignone}
\authornotemark[1]
\authornotemark[2]
\email{renato.mignone@studenti.polito.it}
\affiliation{%
  % \institution{Politecnico di Torino}
  % \city{Turin}
  % \country{Italy}
}

\author{Simone Romano}
\authornotemark[1]
\authornotemark[2]
\email{simone.romano2@studenti.polito.it}
\affiliation{%
  % \institution{Politecnico di Torino}
  % \city{Turin}
  % \country{Italy}
}

% Short author list for page headers
\renewcommand{\shortauthors}{Botticella, Innocenti, Mignone, and Romano}

\date{}

% ---------------------------
% Abstract Section
% ---------------------------
% 00-abstract.tex

\begin{abstract}

    \ldots
    
\end{abstract}


% Create the title block
\maketitle

% Add line numbers for review
\linenumbers

% ---------------------------
% Document Keywords
% ---------------------------
\keywords{LaTeX template, ACM CCS, ACM}

% ---------------------------
% Main Content Sections
% ---------------------------
% 01-background-and-objectives.tex
% ---------------------------

% Section Title
\section{BACKGROUND AND OBJECTIVES} \label{sec:background-and-objectives}

    % Main Content
    
    \ldots
    
% 02-methodology-and-concepts.tex
% ---------------------------

% Section Title
\section{METHODOLOGY AND CONCEPTS}

    % Main Content

    \ldots

    \subsection{Selected Tools}

        \ldots

    \subsection{Network Performance Metrics}
    
        \ldots
        
% 03-experimental-setup-and-test-cases.tex
% ---------------------------

% Section Title
\section{EXPERIMENTAL SETUP AND TEST CASES}

    % Main Content

    \ldots

    \subsection{Equipment and Configuration}

        \ldots

    \subsection{Evaluation Scenarios}

        \ldots
        
% 04-analysis-and-findings.tex
% ---------------------------

% Section Title
\section{ANALYSIS AND FINDINGS} \label{sec:analysis-and-findings}

    % Main Content

    \ldots

    \subsection{TCP Performance} \label{subsec:tcp-performance}

        \ldots

    \subsection{UDP Performance} \label{subsec:udp-performance}

        \ldots

% 05-conclusion.tex
% ---------------------------

% Section Title
\section{CONCLUSION}

    % Main Content

    \ldots
    

% ----------------------------------------
% Appendix
% ----------------------------------------
\clearpage
\appendix
% A-appendix.tex
% ---------------------------

\section{APPENDIX} \label{sec:appendix}

    \ldots

% ----------------------------------------
% Bibliography
% ----------------------------------------
\nocite{*} % Include all references from the .bib file
\bibliographystyle{template/ACM-Reference-Format}
\bibliography{bibliography/references}

\end{document}

%%% Local Variables:
%%% mode: latex
%%% TeX-master: t
%%% End:
