% A-appendix.tex
% ---------------------------

\section{APPENDIX} \label{sec:appendix}

\subsection*{Server Mode Initialization}

\begin{lstlisting}[language=Python, caption={Excerpt for server mode initialization.}]
def run_server():
    """Run iperf3 server with clean output handling."""
    server_logger.info("Starting iperf3 server...")
    
    proc = subprocess.Popen(
        ["iperf3", "-s", "-J"],
        stdout=subprocess.PIPE,
        stderr=subprocess.PIPE,
        text=True,
        bufsize=1,
    )
    # Handle server output and errors in a separate thread...
\end{lstlisting}

\subsection*{Client Mode Execution and Reporting}

\begin{lstlisting}[language=Python, caption={Excerpt for client mode execution.}]
def run_client(server_ip, udp=False, bitrate="1M", iterations=10):
    """Run iperf3 client tests and generate reports."""
    for i in range(iterations):
        cmd = ["iperf3", "-c", server_ip, "-J", "-t", "10", "-i", "1"]
        if udp:
            cmd.extend(["-u", "-b", bitrate])
        
        result = subprocess.run(
            cmd,
            capture_output=True,
            text=True,
            check=True
        )
        
        data = json.loads(result.stdout)
        # Extract test data, compute statistics, and log results...
\end{lstlisting}

\subsection*{Logging and Output Management}

\begin{lstlisting}[language=Python, caption={Excerpt for logging setup.}]
def setup_logger(log_file, name):
    logger = logging.getLogger(name)
    logger.setLevel(logging.DEBUG)
    handler = logging.FileHandler(log_file)
    formatter = logging.Formatter(
        '%(asctime)s - %(levelname)s - %(message)s'
    )
    handler.setFormatter(formatter)
    logger.addHandler(handler)
    return logger
\end{lstlisting}

These snippets illustrate how the script handles server mode initialization, client tests (both TCP and UDP), and logging. Error handling, multithreaded \texttt{stderr} management, and CSV report generation are also included in the complete script.